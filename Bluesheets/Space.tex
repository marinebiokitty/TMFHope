\documentclass[blue]{TMFHope}
\begin{document}
\name{\bHistory{}}

Humanity achieved regular space flight nearly a millennia ago. After colonizing the moon, the next logical step was Mars. On Mars humanity found technology that had certainly not come from earth. Reverse-engineering it took decades, but it led to the ``Jump Drive''. Jump Drives allow ships to move almost instantaneously between any two points in space. The science behind how a Jump works is still being argued among academics, but anyone working on a ship these days knows the practicalities. Jumps are safe and routine, and have allowed humanity to expand far across the Universe. In that time, humanity has found only a few caches of technology like the one on Mars, and precious little is known about the civilization that left it behind.

Expansion into space was not all peaceful. As humanity expanded to the stars, a rift grew between the \pPlan{} and the \pEdge{} that ultimately led to a war 6 years ago. The \pEdge{} wanted to drive ever deeper into space, heedless of caution or cost. As long as exploration was exponential, and there was enough new, inhabitable, planets were found, and of course they didn't encounter anything \emph{dangerous}, their economic plan would theoretically work. The \pPlan{} were more conservative, and wished to have a slower, more cautions exploration, expanding only when humanity needed it. The war was hot and fast, with the \pEdge{} making a series of tactical strikes on key \pPlan{} planets. Over \pHome{}, the planet where the constitutional monarchy of the \pPlan{} is based, the \pPlan{} were forced into a desperate last stand. At first it looked like the \pEdge{} would prevail, but at the last possible second, the \pOld{} jumped in. Against all odds, this single strike ship turned the tide of \pBattle{} and snatched an impossible victory for the \pPlan{}. The morale boost from that single miracle was enough to carry the \pPlan{} to victory against the \pEdge{}.

With the \pPlan{} in charge, expansion is slow, but orderly. Ship are classified as either \pTMN{} (TMN) for military, like the \pOld{}, or \pTMF{} (TMF) for civilian ships, like the \pNew{}. While there are no signs of other civilizations traveling the stars now, humanity maintains hope for first contact some day. Every ship in \pTMF{} is obligated to have a diplomat on crew, just in case the ship makes first contact with a new civilization. There is also a standard rule that romance between crew members in the same chain of command is not allowed. On small \pTMF{} ships, everyone is in the same chain of command. Life on a space ship obligates a hierarchy of command that makes equitable relationships impossible.



\end{document}
