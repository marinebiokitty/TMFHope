\documentclass[blue]{TMFHope}
\begin{document}
\name{\bHope{}}

The \pNew{} is a reliable, unglamorous, merchant ship. She makes her living shuttling raw materials from supply planets to industrial ones, and finished goods back again. She isn't the newest ship on the line, but she is reliable. Her Captain is \cCap{}, an experienced Navy Captain. This is \cCap{\their} first civilian ship after a dishonorable discharge from the Navy.  

The Captain's discharge was contentious, as \cCap{} was Captain of the \pOld{} in \pBattle{}. Technically the \pOld{} saved the day, and indeed the \pPlan{}, but her Captain was hours late in answering the summons. Six years after that battle, a few armchair tacticians still debate whether \pPlan{} had a chance at \pHome{} without whatever miracle the \pOld{} managed. What if the \pOld{} had arrived on time, but had the same odds as any other strike ship of surviving and taking out her target? In polite company those people are usually hushed before they can bring the thought to its logical conclusion.

The ship's Executive Officer (XO), \cXO{} served with \cCap{}, but kept \cXO{\their} job despite the fiasco at \pBattle{} as \cXO{\they} \cXO{\were} not in command. The rest of the crew of the \pNew{} have mild histories by comparison. The crew has all been working together for a few months, except for \cBoy{}, who was taken aboard at the last port the \pNew{} made. All ships, including those in \pTMF{}, has a chain of command to maintain order. The hierarchy matters primarily in determining who is the acting captain if one or more crew members are unfit for command, or if the Captain is on an away mission for \emph{some} reason. \cBoy{} is not considered part of the chain of command, but if \cBoy{\they} were to be the last crew member capable of issuing orders on the \pNew{}, \cBoy{\they} could do so and carry those orders out \cBoy{\themself}. Mostly this means that \cBoy{} has to obey orders from the (acting) captain just like the rest of the crew, but at no point can \cBoy{} \emph{be} the acting captain.

{\bf The \pNew{} Chain of Command:} \emph{this also serves as the Dramatis Personae.}
\begin{enumerate}
  \item \cCap{\full}, \cCap{\they}\textbackslash\cCap{\them} (\cCap{\MYplayer}): The Captain of the \pNew{}. \cCap{\They} \cCap{\are} easy going and likeable. How \cCap{\they} manage\cCap{\plural} to simultaneously be everyone's friend and also maintain appropriate distance and impartiality is quite the trick. \cCap{\They} served in \pBattle{} on the \pOld{}, after which \cCap{\they} \cCap{\were} issued a dishonorable discharge.
  
  \item \cXO{\full}, \cXO{\they}\textbackslash\cXO{\them} (\cXO{\MYplayer}): The Executive Officer of the \pNew{}. \cXO{} served in the \pPlan{} military with \cCap{}. \cXO{\They} left \cXO{\their} post to join the crew of \pNew{}. \cXO{\They} \cXO{\are} a good foil to the Captain, being direct, businesslike, and at times even impatient with crew-members who dawdle at executing orders.
  
  \item \cDip{\full}, \cDip{\they}\textbackslash\cDip{\them} (\cDip{\MYplayer}): The Diplomat assigned to the \pNew{}. \cDip{\They} \cDip{\are} prim and proper, always smiling a smile that doesn't quite reach \cDip{\their} eyes. A consummate diplomat, \cDip{\they} \cDip{\are} always polite, and seem\cDip{\plural} to hold everyone else at arm's length. Still, if you need advice on how to handle a social situation, there is no one better to advise you. In a First Contact situation, the Diplomat is the highest authority and outranks even the Captain.
  
  \item \cMed{\full}, \cMed{\they}\textbackslash\cMed{\them} (\cMed{\MYplayer}): The Doctor on the  \pNew{}. \cMed{\They} \cMed{\are} responsible for keeping the crew whole and hale. \cMed{\They} take\cMed{plural} the responsibility quite seriously, and does a good job looking after the crew. \cMed{} is a little on the quiet side, maybe even awkward among folks \cMed{\they} \cMed{\do}n't know well. In medical emergencies the Doctor is the highest authority and outranks even the Captain.
  
  \item \cEng{\full}, \cEng{\they}\textbackslash\cEng{\them} (\cEng{\MYplayer}): The Engineer who takes care of \pNew{}. \cEng{\They} \cEng{\are} the best of the best, in part due to how long \cEng{\they} \cEng{\have} been an engineer. It's anybody's guess what \cEng{\they} \cEng{\are} doing on a little ship like \pNew{} when \cEng{\they} could have \cEng{\their} pick of the fleet. Still, \cEng{} is the heart of the crew and the \pNew{} wouldn't be the same if \cEng{\they} \cEng{\were} ever to retire.
  
  \item \cWeap{\full}, \cWeap{\they}\textbackslash\cWeap{\them} (\cWeap{\MYplayer}): The Weapons specialist of the \pNew{}. \cWeap{\Theyhave} a busy past, having served in the military, and then transferred to the secret service. Now \cWeap{\they} seem\cWeap{\plural} to just want to do their job minding the \pNew{} laser, usually used to blast asteroids out of the way, and forget \cWeap{\their} past. \cWeap{} get\cWeap{\plural} edgy if you ask too many questions.
  
  \item \cNav{\full}, \cNav{\they}\textbackslash\cNav{\them} (\cNav{\MYplayer}): The Navigator of the \pNew{}. \cNav{\Theyare} in charge of piloting the ship and calculating jumps through hyperspace. While the \pNew{} navigator is fairly new at the job, \cNav{\they} graduated top of \cNav{\their} class and \cNav{are} considered an up and coming talent in the field. \cNav{} is, bubbly and friendly, an open fan\cNav{\kid} of \cCap{}, and an avid history buff. Don't get them started on the history of the war between \pPlan{} and \pEdge{}.
  
  \item \cSci{\full}, \cSci{\they}\textbackslash\cSci{\them} (\cSci{\MYplayer}): The Scientist of the \pNew{}. \cSci{\Theyhave} a comparatively easy job on the ship, mostly just hanging around and being an extra pair of hands when things are running normally. If the \pNew{} runs into anything \emph{weird} though, a good scientist can be the difference between surviving and being space debris. \cSci{\Theyare} a good-natured, rambunctious sort, and while \cSci{\they} claim not to look for fights, they sure seem to spend a lot of their port-leave in the infirmary after one ``jolly fight'' or another.
  
  \item \cBoy{\full}, \cBoy{\they}\textbackslash\cBoy{\them} (\cBoy{\MYplayer}): The newest member of the \pNew{}. \cBoy{\Theywere} picked up at the last port of call for the \pNew{}, where \cBoy{\they} struck up a friendship with \cEng{} and \cNav{}. \cBoy{\Theyare} picking up life aboard \pNew{} quickly, but still seem shy and nervous about messing up.
\end{enumerate}

The current predicament of the \pNew{} began when she tried to make a routine ``Jump'' on a normal route. Something went wrong in one of the systems, and somehow the ship got\ldots{} stuck?\ldots{} in the middle of the Jump. While a Jump does not involve inertia on any of the standard 3 spatial dimensions, there is nevertheless momentum, and it's sudden absence caused every crew member to pass out briefly. Upon regaining consciousness, the crew have found themselves in strangely monochromatic versions of the ship, with access to only a few rooms. As they look around, they find only a few other crew members are present with them. Where is the rest of the ship? and the rest of the crew? What happened? And how can they fix it? That much is obvious - the ship must be repaired, and whatever happened reversed, and quickly. Whatever happened in this botched Jump is putting excessive strain on the ship. It can only stand another 2 hours of this before it will fall apart. No one wants to find out what dying mid-jump is like.

\end{document}
