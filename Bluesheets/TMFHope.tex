\documentclass[blue]{TMFHope}
\begin{document}
\name{\bHope{}}

The \pNew{} is a reliable, unglamorous, merchant ship. She makes her living shuttling raw materials from supply planets to industrial ones, and finished goods back again. She isn't the newest ship on the line, but she is reliable. Her Captain is \cCap{}, an experienced Navy Captain. This is \cCap{\their} first civilian ship after a dishonorable discharge from the Navy.  

The Captain's discharge was contentious, as \cCap{} was Captain of the \pOld{} in \pBattle{}. Technically the \pOld{} saved the day, and indeed the \pPlan{}, but her Captain was hours late in answering the summons. Six years after that battle, a few armchair tacticians still debate whether \pPlan{} had a chance at \pHome{} without whatever miracle the \pOld{} managed. What if the \pOld{} had arrived on time, but had the same odds as any other strike ship of surviving and taking out her target? In polite company those people are usually hushed before they can bring the thought to its logical conclusion.

The ship's Executive Officer (XO), \cXO{} served with \cCap{}, but kept \cXO{\their} job despite the fiasco at \pBattle{} as \cXO{\the} were not in command. The rest of the crew of the \pNew{} have mild histories by comparison. The crew has all been working together for a few months, except for \cBoy{}, who was taken aboard at the last port the \pNew{} made. All ships, including those in \pTMF{}, has a chain of command to maintain order. The hierarchy matters primarily in determining who is the acting captain if one or more crew members are unfit for command, or if the Captain is on an away mission for \emph{some} reason. \cBoy{} is not considered part of the chain of command, but if \cBoy{\they} were to be the last crew member capable of issuing orders on the \pNew{}, \cBoy{\they} could do so and carry those orders out \cBoy{\themself}. Mostly this means that \cBoy{} has to obey orders from the (acting) captain just like the rest of the crew, but at no point can \cBoy{} \emph{be} the acting captain.

{\bf The \pNew{} Chain of Command:}
\begin{enumerate}
  \item The Captain: \cCap{}
	\item The Executive Officer (XO): \cXO{}
	\item The Diplomat: \cDip{} - In a First Contact situation, the Diplomat is the highest authority and outranks even the Captain.
	\item The Doctor: \cMed{} - In medical emergencies the Doctor is the highest authority and outranks even the Captain.
	\item The Engineer: \cEng{}
	\item The Weapons Officer: \cWeap{}
	\item The Navigator: \cNav{}
	\item The Science Officer: \cSci{}
\end{enumerate}

The current predicament of the \pNew{} began when she tried to make a routine ``Jump'' on a normal route. Something went wrong in one of the systems, and somehow the ship got\ldots{} stuck?\ldots{} in the middle of the Jump. While a Jump does not involve inertia on any of the standard 3 spatial dimensions, there is nevertheless momentum, and it's sudden absence caused every crew member to pass out briefly. Upon regaining consciousness, the crew have found themselves in strangely monochromatic versions of the ship, with access to only a few rooms. As they look around, they find only a few other crew members are present with them. Where is the rest of the ship? and the rest of the crew? What happened? And how can they fix it? That much is obvious - the ship must be repaired, and whatever happened reversed, and quickly. Whatever happened in this botched Jump is putting excessive strain on the ship. It can only stand another 2 hours of this before it will fall apart. No one wants to find out what dying mid-jump is like.

\end{document}
