\documentclass[char]{TMFHope}
\begin{document}
\name{\cWeap{}}

Your name is \cWeap{\full}.  It's not that you like violence, it's that it seems to follow you around no matter what you do. Running Weapons for a TMF ship is quiet. Once in a handful of missions you'll be called on to blast a few asteroids out of the way. It's kind of a boring life, but after what happened in the space above \pHome{}, it's all you can manage. Anything more immediate sets off the flashbacks. Life on \pNew{} pays the bills. The quiet life is good for you.

You started your career as a security guard, but that was awfully repetitive - you wanted more excitement. So you joined the marines. You served on the TMN Leviathan during \pBattle{}. That ship was one of the \pPlan{}' main battleships. As a marine, you spent most of your time in a small dog-fighting ship, trying to protect her - not what you were trained for, but the \pPlan{} was caught flat footed by the attack and was making a lot of desperate calls to try to survive. Despite your team's every attempt, she was torn to pieces over the course of several hours. It was absolute torture to watch section after section vent into space, seeing crew-members sucked out into the cold, dark, vacuum of space, hearing your teammates dying screams through your headset. Everything about it haunts you. War is not glorious, and anyone who says so is an idiot, and thinks their audience is too.

Your opinions on the \pOld{} and her Captain, \cCap{} are mixed at best. On the one hand, yes that ship turned the tide of the battle, and that probably saved your life personally. But on the other hand, if she had been on time, how many of your friends in dog fighting ships, and how many good crew-members back on the Leviathan would have survived too? There was never a satisfactory answer in the public dossier as to why \pOld{} was late. It's no coincidence that you are serving on the \pNew{}. You picked this ship to apply to because of her Captain. You keep waiting for the right moment to confront \cCap{} about what happened. The moment is never right. If you handle it wrong, you could be fired, and you've rather come to like life on the \pNew{}. But damned if you are willing to take that question to the grave unanswered. Where was the \pOld{}? Why didn't she come when she was called to \pBattle{}?

After the war, you tried the \pPlan{} secret service. You became a bodyguard. It was a good living and you enjoyed it for the most part. The small potential for danger gave you the vibrancy you craved in a manageable dose. You were ultimately assigned to a diplomat on \pHome{}, \cDip{\full}. \cDip{\They} \cDip{\were} friendly and gregarious, but surprisingly slippery and hard to keep track of - which made your job more interesting. And then everything went sideways. Three years ago, \cDip{} was supposed to meet with a representative from a violent splinter group. You had been warned that the whole thing might be a trick - so when the representative and their three (3!) guards all reached into their coats at the same time and started to draw guns, you did what you had to to protect \cDip{}. You have no idea how you two survived that shoot out (which parts were real, and which parts were \pBattle{} all over again, you'll never know), but you got \cDip{\them} to safety. Now you re-live that terrible day over and over. And \cDip{} has never even shown you a smidgen of gratefulness for saving \cDip{\their} life.

Crowds make you nervous. Loud noises make you jump. Sudden movements make you angry. Life on \pNew{} is a good pace. A slow one. You have friends here. Your therapist says having friends is good for you. \cSci{} is always down to spar, and \cNav{} is a surprisingly good listener. \cNav{\They} \cNav{\have} been helping you with the PTSD (mostly nightmares and flashbacks). It was even getting easier to manage for a while - until today. About an hour before the jump, you started to see apparitions. What used to be ghosts that haunted your dreams were now wandering around the \pNew{}.

There are 2 people aboard who are definitely \emph{not} your friend though: \cDip{} of course, and \cEng{\full}. \cEng{\They} \cEng{\are} old and crotchety and definitely hiding something. \cEng{} never lets you hang out in the engine room, and \cEng{\they} \cEng{\are} spending less and less time around the rest of the crew. It's so weird that \cEng{\they} supposedly connected with \cBoy{} so quickly at the last port. Maybe they are in cahoots for something? Something like this failed jump\ldots{} Not that you have any idea what \cEng{} has to gain, but that's just another piece you intend to uncover. This would all be much easier if the damn intercoms worked. Tech isn't your specialty, but you've got this old \iWT{}\ldots Maybe you can come up with something clever?

\subsection*{Abilities}
\begin{tabular}{|p{3cm}|p{1.5cm}|p{6.5cm}|p{5cm}|} 
 \hline
 \textbf{Ability Name} & \textbf{Uses} & \textbf{Ability Details} & \textbf{Ability Effect (what others see)} \\ 
\hline 
 \aThreat{\MYname} & Unlimited & \aThreat{\MYtext} & \aThreat{\MYeffect} \\ 
\hline
 \aFirstAid{\MYname} & 3 uses & \aFirstAid{\MYtext} & \aFirstAid{\MYeffect}\\ 
 \hline
	\aDefense{\MYname} & Unlimited & \aDefense{\MYtext} & \aDefense{\MYeffect} \\ 
\hline
	\aPractice{\MYname} & Unlimited & \aPractice{\MYtext} & \aPractice{\MYeffect} \\ 
\hline
\end{tabular}

\subsection*{Memory and Event Packets}
\begin{enumerate}
	\item \mPractice{\MYname}
	\item \mWAlpha{\MYname}
	\item \mWeaponsOne{\MYname}
	\item \mWeaponsTwo{\MYname}
	\item \mWeaponsThree{\MYname}
	\item \mWClosure{\MYname}
	\item \mThreatOne{\MYname}
	\item \mThreatTwo{\MYname}
	\item \mThreatThree{\MYname}
	\item \mThreatFour{\MYname}
	\item \mBroom{\MYname}
	\item \mLab{\MYname}
	\item \mPatient{\MYname}
	\item \mKitchen{\MYname}
	\item \mWeight{\MYname}
	\item \mTheater{\MYname}
	\item \mCrates{\MYname}
\end{enumerate}

\begin{itemz}[Goals]
	\item Make \cDip{} look bad. You've had enough of \cDip{\their} holier-than-thou attitude. Maybe this is your chance to get rid of \cDip{\them}.
	\item Get rid of the apparitions somehow! Maybe closure on the \pBattle{} will help?
	\item Convert the walkie talkie to something that will restore ship-wide communication.
	\item Figure out why \cEng{} has been acting furtive.  Maybe \cEng{\they} caused this disaster?
\end{itemz}

\begin{itemz}[Notes]
	\item You start game in the Science Lab in the ``Cyan'' dimension. You also have access to the Cafeteria in the ``Blue'' dimension
	\item The two dimensions you can access {\em feel} different somehow -- cyan feels purer, more primary, while blue feels half-real, halfway between other dimensions, more secondary.  Is that how everyone else feels?
	\item To try to reinstate ship-wide communications start with the \iWT{\MYname} in your possession. Spend 2 minutes in a game space, by yourself, fiddling with the walkie talkie. At the end of this time, go to the ``mechanics-only'' channel, and use the ``searchcontainer'' command for ``Packet 2''. Take the ``Modified Walkie Talkie'' item from the packet. Destroy the \iWT{\MYname} using the ``consume" item Since \cBoy{} worked on a switch board on the station where they grew up, maybe \cBoy{\they} have an idea what to try next? Pass the ``Modified Walkie Talkie'' item along to \cBoy{}.
\end{itemz}

\begin{contacts}
	\contact{\cDip{}} The diplomat whose life you saved. Eternally \textbf{ungrateful.}
	\contact{\cNav{}} A friend who has been helping you manage your PTSD.
	\contact{\cCap{}} The Captain of the \pOld{}. You need to know why she was late to the battle of \pHome{}.
	\contact{\cEng{}} The ship's engineer. \cEng{\They} \cEng{\have} been increasingly more furtive and suspicious; the more so since the \pNew{} took on \cBoy{}.
\end{contacts}

\end{document}
