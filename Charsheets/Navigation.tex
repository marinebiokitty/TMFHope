\documentclass[char]{TMFHope}
\begin{document}
\name{\cNav{}}

The \pNew{} is your dream job. Imagine how lucky you are to have landed it as your {\em first} job too. Someone looking at your grades coming out of the prestigious flight academy \pHome{} has to offer might be surprised that someone as good as you would want to work on a dinky little merchant ship like \pNew{}. They would, of course, completely fail to understand the perks.

First is Captain \cCap{\full}. You have been a fan\cNav{\kid} of \cCap{\theirs} since you were small. What \pOld{} did in \pBattle{} was something else. Something impossible. And what did \cCap{} get? A dishonorable discharge? You are still indignant over the mistreatment of your hero. Working on the same ship with \cCap{\them}, you might finally be able to achieve one of your dreams - to find out what really happened that fateful day, and clear the Captain's name.

Secondly is \cBoy{\full}. You met while trolling the internet some years back, and became pen pals. Occasional emails became messages exchanged daily and soon grew into a close friendship. From there, it became something else - not romance - but an unmistakable desire to spend your life with \cBoy{\them}. By the time you graduated, you were committed to finding a way to get \cBoy{} off the backwater space-port \cBoy{\they} \cBoy{\were} stuck on, so you could build a future together. As luck would have it the \pNew{} scheduled a route through \cBoy{\their} home port. So you and \cBoy{} hatched a plan to get \cBoy{\them} hired on as a deck hand. It was a dicey thing there for a minute. While \cEng{} hit it off with \cBoy{} immediately, \cEng{\their} endless dislike for you nearly scuttled the operation. Still, all's well that end's well.

Thirdly is \cWeap{}. \cWeap{\They} \cWeap{\are} an unexpected perk -- well, maybe perk is the wrong word. \cWeap{} has had a hard life, and is clearly suffering from PTSD. You do your best to support \cWeap{\them} between \cWeap{\their} therapy sessions, mostly by listening and helping \cWeap{\them} ground when they they have flashbacks. \cWeap{} has been having many fewer such experiences recently, and you are both hopeful about the progress.

You suppose no situation can be perfect though, the ship's engineer, \cEng{} has not liked you since the day you set foot on the \pNew{}. You can't understand why. You like ships. \cEng{\They} like ships. The two of you should be the best of friends, chatting away endless hours speculating on wonderful theoretical and practical improvements to ships and space travel in general.

Back to the positives, and actually maybe most importantly, you've had time to develop your pet theory. If you hadn't wanted to be a pilot so bad, and such a perfect opportunity hadn't come up, you might have ended up an academic. You have a pet theory about ``jump'' technology, and the moderate down time on \pNew{} has given you time to complete the calculations to back it up - at least in theory. With so few caches of the technology that led to the jump drive, you believe the sentient race that left it must have abandoned this dimension for another, and left behind clues with the hopes that humanity could follow them. Who better to figure out how to access new dimensions than someone who can calculate jumps in their head? But theory can only take you so far. Last week, you started making modifications to the \pNew{}'s engine's and navigation systems, in preparation to jump the ship to a new dimension and prove your theory. Maybe they'll give you an honorary PhD when it works\ldots

In retrospect, maybe you should have asked the Captain first. Or had someone check your calculations. Or published a paper on the theory to solicit peer review. Because something definitely went wrong! How to fix it? How to not lose your job over it? Could you blame someone else? And can you possibly learn anything from this that will allow you to figure out how to do it right next time?

\subsection*{Abilities}
\begin{tabular}{|p{3cm}|p{1.5cm}|p{8.5cm}|p{3cm}|} 
 \hline
 \textbf{Ability Name} & \textbf{Uses} & \textbf{Ability Details} & \textbf{Ability Effect (what others see)} \\ 
\hline 
 \aCalculate{\MYname} & Unlimited & \aCalculate{\MYtext} & \aCalculate{\MYeffect} \\
\hline 
 \aInvestigate{\MYname} & Unlimited & \aInvestigate{\MYtext} & \aInvestigate{\MYeffect} \\
 \hline
\end{tabular}

\subsection*{Memory and Event Packets}
\begin{enumerate}
	\item \mPractice{\MYname}
	\item \mAlpha{\MYname}
	\item \mNavOne{\MYname}
	\item \mNavTwo{\MYname}
	\item \mNavThree{\MYname}
	\item \mNavFour{\MYname}
	\item \mNavFive{\MYname}
	\item \mBroom{\MYname}
	\item \mLab{\MYname}
	\item \mPatient{\MYname}
	\item \mKitchen{\MYname}
	\item \mWeight{\MYname}
	\item \mTheater{\MYname}
	\item \mCrates{\MYname}
\end{enumerate}

\begin{itemz}[Goals]
	\item Learn all you can about this weird situation so you can make it possible for ships to ``jump'' to a new dimension.
	\item Find out what happened at \pBattle{} so you can clear \cCap{}'s name.
	\item Get someone else blamed for what happened to the ship so you can keep your job.
	\item Support \cWeap{}. This is probably very stressful for \cWeap{\them}.
	\item Protect \cBoy{} and figure out if you can placate \cEng{} somehow.
\end{itemz}

\begin{itemz}[Notes]
	\item You start game in the Science Lab in the ``Cyan'' dimension. You also have access to the Storage Bays in the ``Black'' dimension.
	\item The two dimensions you can access {\em feel} different somehow -- cyan feels purer, more primary, while black feels half-real, halfway between other dimensions, more secondary.  Is that how everyone else feels?
\end{itemz}

\begin{contacts}
	\contact{\cCap{}} Your hero, who was wrongfully punished after \pBattle{}. You intend to clear \cCap{\their} name.
	\contact{\cBoy{}} Your platonic life partner. You helped get \cBoy{\them} a job on \pNew{}.
	\contact{\cWeap{}} A relatively new friend.
	\contact{\cEng{}} Why \cEng{\does}n't \cEng{\they} like you?
\end{contacts}

\end{document}
