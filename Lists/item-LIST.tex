
%%
%% This file creates the Item, ItemPacket, ItemFold, ItemEnvelope, and
%% ItemLabel datatypes, and creates macros for each.  These are for
%% various types of in-game items.
%%
%%%%%


%%%%%
%% Item macros are for normal item cards.
\DECLARESUBTYPE{Item}{TransElement}
\PRESETS{Item}{
  \FD\MYtext	{} %% longer text of item
  \FD\MYmark	{} %% possible contents of shaded ``mark'' on card
  \FD\MYbulky	{0} %% potential bulkiness
  \FD\MYcapacity{N/A} %% potential capacity
  \sd\MYlistmap	{\item\MYname\ifx\MYnumber\empty\else\ (\MYnumber)\fi}
  }


%%%%%
%% \prop
%% \unstash
%% \bulky{<number>}
%% \contain{<number>}
%%
%% \prop inside an Item macro labels the card as a prop.  \unstash
%% labels the card as unstashable.  \bulky{n} labels the card as
%% n-hands bulky.  \contain{n} labels the card with n-hands capacity.
\def\prop{%
  \append\MYmark{ ~PROP~ }}
\def\unstash{%
  \append\MYmark{ ~UNSTASHABLE~ }}
\def\bulky#1{%
  \s\MYbulky{#1}%
  \append\MYmark{\mbox{ ~\MYbulky-Hand~Bulky~ }}}
\def\contain#1{%
  \s\MYcapacity{#1}%
  \append\MYmark{\mbox{ ~\MYcapacity-Hand~Capacity~ }}}


%%%%%
%% ItemPacket macros are for item cards with an attached packet.
%% They are a subtype of Item.
\DECLARESUBTYPE{ItemPacket}{Item}
\PRESETS{ItemPacket}{
  \F\MYcontents
  }


%%%%%
%% ItemFold macros are for items represented by just a folded packet.
%% They are a subtype of ItemPacket, with the longer text and ``mark''
%% left blank, since they have no actual item card.
\DECLARESUBTYPE{ItemFold}{ItemPacket}
\PRESETS{ItemFold}{
  \s\MYmark{}
  }


%%%%%
%% ItemEnvelope macros are for items represented by just an envelope.
%% They are a subtype of ItemPacket, with the longer text and ``mark''
%% left blank, since they have no actual item card.
\DECLARESUBTYPE{ItemEnvelope}{ItemPacket}
\PRESETS{ItemEnvelope}{
  \s\MYmark{}
  }


%%%%%
%% ItemLabel macros are for small labels that would get used on
%% physreps, e.g. gun labels.  The ``mark'' is left blank, since
%% it isn't used for these.
\DECLARESUBTYPE{ItemLabel}{Item}
\PRESETS{ItemLabel}{
  \s\MYmark{}
  }


%%%%%
%% \icard[<extras>]{<name>}{<number>}{<text>}
%% \specialicard[<extras>]{<name>}{<number>}{<text>}{<mark>}
%% \itempacket[<extras>]{<name>}{<number>}{<text>}{<mark>}{<contents>}
%% \itemfold{<name>}{<number>}{<text>}{<contents>}
%% \itemenvelope{<name>}{<number>}{<text>}{<contents>}
%% \itemlabel{<name>}{<number>}{<text>}
%%
%% These are wrappers around \INSTANCE, useful for 1-shots.
%%
%% For \icard, \specialicard, and \itempacket, the optional <extras>
%% (in []'s) is for things like \unstash and \bulky{3}.  For example,
%% \icard[\prop\contain{2}]{..}{..}{..}{..} gives an item that has a
%% prop and 3-hands capacity.
%%
%% The last arg (#5) to \specialicard is for anything extra you may
%% want in the ``mark''
\newinstance{Item}{\icard[4][]}{
  \s\MYname{#2}\s\MYnumber{#3}\s\MYtext{#4}#1}
\newinstance{Item}{\specialicard[5][]}{
  \s\MYname{#2}\s\MYnumber{#3}\s\MYtext{#4}\s\MYmark{#5}#1}
\newinstance{ItemPacket}{\itempacket[6][]}{
  \s\MYname{#2}\s\MYnumber{#3}\s\MYtext{#4}\s\MYmark{#5}\s\MYcontents{#6}#1}
\newinstance{ItemFold}{\itemfold[4]}{
  \s\MYname{#1}\s\MYnumber{#2}\s\MYtext{#3}\s\MYcontents{#4}}
\newinstance{ItemEnvelope}{\itemenvelope[4]}{
  \s\MYname{#1}\s\MYnumber{#2}\s\MYtext{#3}\s\MYcontents{#4}}
\newinstance{ItemLabel}{\itemlabel[3]}{
  \s\MYname{#1}\s\MYnumber{#2}\s\MYtext{#3}}


%%%%%%%%%%%%%%%%%%%%%%%%%%%%%%%%%%%%%%%%%%%%%%%%%%%%%%%%%%%%%%%%%%

\NEW{Item}{\iTest}{
  \s\MYname	{Test Item}
  \s\MYnumber	{0000}
  \s\MYtext	{A Test Item Card}
  }

\NEW{ItemPacket}{\iTestPacket}{
  \s\MYname	{Test Item}
  \s\MYnumber	{0000}
  \s\MYtext	{A Test Item with a big red button.  Open packet if
		you press the big red button.}
  \s\MYcontents	{The item beeps at you.}
  }

\NEW{ItemFold}{\iTestFold}{
  \s\MYname	{Test Food}
  \s\MYnumber	{0000}
  \s\MYtext	{open if you eat}
  \s\MYcontents	{It tastes yummy.}
  }

\NEW{ItemEnvelope}{\iTestEnvelope}{
  \s\MYname	{Test Food}
  \s\MYnumber	{0000}
  \s\MYtext	{open if you eat}
  \s\MYcontents	{It tastes yummy.}
  }

\NEW{ItemLabel}{\iTestLabel}{
  \s\MYname	{Test Gun Label}
  \s\MYnumber	{0000}
  \s\MYtext	{Disc gun, loadable to 20 shots.}
  }

\NEW{Item}{\iWhatzit}{
  \s\MYname	{Whatzit}
  \s\MYnumber	{12345}
  \s\MYtext	{If you press it, open packet a.  If you twirl it, open
		packet b.  If you pull it, open packet c.}
  \bulky	{1}
  \s\MYsigns	{\signstrip{a}{it goes ``beep.''}
		\signstrip{b}{it goes ``whoop.''}
		\signstrip{c}{it goes ``bang.''}
		}
  \s\MYabils	{\ability{Stop Crying}{By futzing with the Whatzit, you
		can make babies stop crying.}{I make the baby stop
		crying.}
		}
  }


%%%%%%%%%%%%%%%%%%%%%%%%%%%%%%%%%%%%%%%%%%%%%%%%%%%%%%%%%%%%%%%%%%
%%Repair items
\NEW{Item}{\iTorch}{
  \s\MYname	{Welding Torch}
  \s\MYnumber	{0000}
  \s\MYtext	{A welding torch. This item has a \textbf{5 minute cooldown} that begins when the task is completed before it can be used for a new task. \emph{OOC Note: Players are responsible for keeping track of this cooldown and passing on that information if the item is handed off to someone else.}}
  }
  
\NEW{Item}{\iPutty}{
  \s\MYname	{Patching putty}
  \s\MYnumber	{0000}
  \s\MYtext	{Putty used to patch holes or occupy one’s hands}
}

\NEW{Item}{\iWires}{
  \s\MYname	{Electrical Wires}
  \s\MYnumber	{0000}
  \s\MYtext	{A collection of electrical wires.}
}

\NEW{Item}{\iRope}{
  \s\MYname	{Resistance Training Bands}
  \s\MYnumber	{0000}
  \s\MYtext	{Brightly colored resistance bands for strength training. Could be used to tie something down in a pinch. (OOC Note: These can be used as rope to restrain someone.)}
}

\NEW{Item}{\iRivets}{
  \s\MYname	{A handful of rivets}
  \s\MYnumber	{0000}
  \s\MYtext	{A handful of rivets; typically used to repair kitchen equipment.}
}

\NEW{Item}{\iBook}{
  \s\MYname	{A Book on Ship Repair}
  \s\MYnumber	{0000}
  \s\MYtext	{Having this book in your possession grants you the “engineering” ability for as long as you have the book with you. “Engineering” = You can repair parts of the ship, assuming you and any assistants have the listed necessary supplies. This activity takes 6 minutes, but having a second person helping (they do not need this ability) cuts the time in half (to 3 minutes). During this time, everyone helping must stay in that location, and may not engage in other activities (i.e.: reading other signs, engaging in combat, etc.) If either you or your assistant engage in another activity, the timer starts over. At the end of the time, consume the items that need to be consumed (The sign will tell you which ones), and switch signs (the signs will tell you which ones.)}
}

\NEW{Item}{\iSander}{
  \s\MYname	{Electric Sander}
  \s\MYnumber	{0000}
  \s\MYtext	{An electric sander. It is nearly worn out.}
}

\NEW{Item}{\iPlate}{
  \s\MYname	{Large Metal Plate}
  \s\MYnumber	{0000}
  \s\MYtext	{A large sheet of metal plating, used for various ship repairs.}
  \s\\MYbulky	{2}
}

%% The Scanners
\NEW{Item}{\iMedScan}{
  \s\MYname	{Handheld Medical Scanner}
  \s\MYnumber	{0000}
  \s\MYtext	{A handheld medical scanner that can only be operated by a trained medical professional}
}

\NEW{Item}{\iSciScan}{
  \s\MYname	{Handheld Science Scanner}
  \s\MYnumber	{0000}
  \s\MYtext	{A handheld science scanner that can only be operated by a trained science officer}
}



%%Transporter
\NEW{Item}{\iCrystal}{
  \s\MYname	{Teleporter Power Crystal}
  \s\MYnumber	{0000}
  \s\MYtext	{An inert green crystal. These are very rare, very expensive, and used to power the ship’s teleporter.}
  }


%%Items starting in inventory
\NEW{Item}{\iKey}{
  \s\MYname	{Authorization Key}
  \s\MYnumber	{0000}
  \s\MYtext	{An authorization key that allows for unorthodox use of ship's systems, such as point to point teleport within the ship.}
  }
	
\NEW{Item}{\iWT}{
  \s\MYname	{Regular Walkie Talkie}
  \s\MYnumber	{0000}
  \s\MYtext	{One regular Walkie Talkie. Not useful in normal circumstances, but maybe it can be modified somehow?}
  }
	
\NEW{Item}{\iVoice}{
  \s\MYname	{Voice Recorder}
  \s\MYnumber	{0000}
  \s\MYtext	{A handheld voice recorder. You may at any time declare that you are recording, and record a conversation for posterity. While the recorder has way more memory than you could fill in two hours, the encoding on previous recordings means that only the individual who made a recording will know which one it is to play it back without taking several hours to decipher it all. (OOC: you will need to keep track of what you have recorded, and only you can play back a recording. i.e.: you cannot record a command from the captain, and then hand the recorder off to person 3 to play it back for person 4. This is a kludge for play-ability.)}
  }

%%%%%%%%%%%%%%%%%%%%%%%%%%%%%%%%%%%%%%%%%%%%%%%%%%%%%%%%%%%%%%%%%%
