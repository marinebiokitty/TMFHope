\documentclass[green]{TMFHope}
\parindent=0pt
\begin{document}
\name{\gDiscord{}}

``TMF Hope'' introduces a new technology in the form of a Discord Bot to manage some of the mechanics of game. We \textbf{request} all players to invest some time in familiarizing themselves with the commands \textbf{before game}. The syntax is very important. Incorrect capitalization, punctuation, and spacing can all cause the Bot to fail. The more familiar you are with using the commands ahead of time, the less often you'll type something in wrong. To facilitate this, once you are in the Discord server, you should go through the exercise below. If you can do it \textbf{before} game, it will be to your benefit. Parts of the exercise will list exact commands to use, other parts will challenge you to work out what the proper command should be based on context and previous actions you've taken. This exercise intentionally repeats some commands to give you more practice with them. The whole exercise shouldn't take you more than 15 minutes.

\begin{enumerate}
  \item Change your nickname to your character name and pronouns; your rank on the ship optional. Keep in mind Discord has a 32 character limit on names. I.E. Cpt Fiona Calloway (she). If you have room, feel free to append your player name and pronouns.
  \item Navigate to the \textbf{``add-roles''} channel in the OOC category.
	\begin{enumerate}
  \item React to the \textbf{first} message there with the appropriate emoji to pick up your role on the ship. \textbf{you must do this first, otherwise most of the rest of this exercise won't work.} Once you add your role you should see a text channel matching your role under ``private channels.''
	\item React to the \textbf{second} message with the \textbf{2 emojis} that match the \textbf{2} areas of the ship your character has access to. These are listed in your character sheet. If you are confused, ``@gm'' and we'll come help as soon as we are able.
	\item React to the \textbf{third} message to pick up the ``player'' role. This won't change anything right now, but may be important during game.
  \end{enumerate}
	\item Navigate to your PRIVATE CHANNEL. Check the pinned message and make sure the stats listed there match the ones at the bottom of your character sheet.
	\item In your private channel, type \textbf{``?inventory''} to see your items, abilities, and mem packets (by name). Do NOT react to delete the message.
	\item \textbf{PLEASE compare these lists to what is on your character sheet.} Let us know ASAP if anything doesn't match. Doing this BEFORE game will help your game experience go more smoothly. Due to the way the Bot loads in content, the list in Discord may not be in the same order as the list on your character sheet. Please check carefully.
	\item In your private channel, type ``?abilities'' to review the details of your abilities. It should be the same information you have on your character sheet.
  \item Trigger your practice mem packet by typing ``?triggerhere Practice'' into your private channel. The bot will print a message back in this same channel with the contents of the packet.
  \item Navigate to the \textbf{``Playground''} channel in the OOC category.
  \item Type \textbf{``?search''} into the message box and press enter. Familiarize yourself with the output that the discord  bot prints in the channel. You can always repeat this command if you forget the name of a sign or container.
	\item Read the sign by typing \textbf{``?readsign Sign 1''} and press enter. Read what the discord bot prints in the channel. Notice that this sign says you can switch it with another sign called ``Sign 2.''
	\item Switch an active and inactive sign by typing \textbf{``?changesign Sign 1 . Sign 2''}. 
	\item Search the area again. Notice that Sign 1 is not listed any more, but Sign 2 is.
	\item Read ``Sign 2'' using the appropriate command.
	\item Switch the signs back again using the appropriate command.
	\item Go to your PRIVATE CHANNEL: type \textbf{``?inventory''} to see your items, abilities, and mem packets (by name). Do NOT react to delete the message.
	\item Return to the ``Playground'' channel.
	\item Read the container by typing \textbf{``?searchcontainer Box''}. Read what the discord bot prints in the channel. React \textbf{``X''} to end the interaction and delete the message.
	\item Read the container again. This time search the container by reacting \textbf{``Check mark''}.
	\item Wait for the search to occur and the bot to start printing items. Different containers may have different search times; this Box has a 10 second search time. It will take a few seconds for the bot to print everything once it starts. You will know it is done when it prints the final option ``React here to not take any items.''
	\item React with the \textbf{``check mark''} to one of the blobs to take it.
	\item Go back to your PRIVATE CHANNEL: Check your inventory using the appropriate command. Compare the item list from before to the one you have now. You should see the blob you just took. 
	\item Check the details of the Blob by typing ``?listitem Blob'' in your private channel. This is how you can read the description of items you have. Please try to do this with each item you pick up. Some items will have important information in the description that can only be seen by using the ``?listitem'' command.
	\item Return to the ``Playground'' channel.
	\item Put the blob back in the container by typing \textbf{``?drop Blob . Box''}.
	\item Go back to your PRIVATE CHANNEL: Check your inventory using the appropriate command. Compare the item list from before to the one you have now. You should see the blob is gone. Delete the most recent inventory by reacting ``check mark''. Try it with the older messages and notice that it doesn't work if those messages are more than 10 seconds old.
	\item Take a blob from the box again using the appropriate command.
	\item Give the blob to another character who is on the server with you at the same time by typing \textbf{``?give $<$@user$>$ Blob''}. You can give the blob to the GM by listing ``@GM: Acata (she/her)'' as the $<$@user$>$ if no one else is around. NOTE: You cannot tag a role for this to work (i.e: GM, Captain, etc). You can only tag users by the nickname they are using in the server (aka: their character name) or their Discord User-name.
	\item Have them give the blob back to you using the appropriate command.
	\item Consume your blob by typing \textbf{``?consumeitem Blob''}. This will remove the blob from game. If you check your inventory one more time, you'll see that the blob is gone.
	\item Use your practice ability in the ``Playground'' channel by typing ``?useability $<$@user$>$ Practice''. You should see a message pop up that tags the user and prints the Ability Effect in the channel. You can use the ability on the GM by listing ``@GM: Acata (she/her)'' as the $<$@user$>$ if no one else is around. Not all abilities will require a user, so don't forget to use ``?abilities'' in your private channel if you need to check whether an ability requires a user or not. It will tell you in the description: ``you must @ a user to use this ability,'' if one is necessary.
\end{enumerate}

Here is a summary of the ground rules for game, some reminders about how the Discord Bot will work, and how Discord interacts with this game.
\begin{itemize}
	\item Please don't take any actions in any game spaces on the server, or open any mem packets not instructed above, before ``game on'' is called. Feel free to continue to mess around in the Playground.
	\item Please don't share your character's secret(s) before game.
	\item You cannot DM other players as part of game communication. Communication is pretty strictly restricted in this game, for many reasons, including game balance. So please don't send each other DMs.
	\item To move between spaces you have access to, switch both voice and text channels. We find fewer audio issues if you disconnect from one voice channel before connecting to the new one.
	\item You can always reach a GM by tagging ``@GM'' in whichever channel you want the GM to come to.
\end{itemize}
\end{document}
