\documentclass[green]{TMFHope}
\parindent=0pt
\begin{document}
\name{\gDiscord{}}

``TMF Hope'' introduces a new technology in the form of a Discord Bot to manage some of the mechanics of game. We \textbf{strongly encourage} all players to invest some time in familiarizing themselves with the commands. The syntax is very important. Incorrect capitalization, punctuation, and spacing can all cause the Bot to fail. The more familiar you are with using the commands ahead of time, the less often you'll type something in wrong. To facilitate this, once you are in the Discord server, you should go through the exercise below. Parts of the exercise will list exact commands to use, other parts will challenge you to work out what the proper command should be based on context and previous actions you've taken. This exercise intentionally repeats some commands to give you more practice with them. The whole exercise shouldn't take you more than 15 minutes.

\begin{enumerate}
  \item Navigate to the \textbf{``add-roles''} channel in the OOC category.
  \item React to the message there with the appropriate emoji to pick up your role. \textbf{you must do this first, otherwise most of the rest of this exercise won't work.} Once you add your role you should see a text channel matching your role under ``private channels,'' and 2 game spaces that match the ones listed in your character sheet.
  \item Navigate to your PRIVATE CHANNEL: type \textbf{``?inventory''} to see your items, abilities, and mem packets (by name). Do NOT react to delete the message. 
  \item Trigger your practice mem packet by typing ``?trigger Practice'' into your private channel. \textbf{you will receive a DM from the Bot with the contents of the packet.} The message will NOT post to your private channel.
  \item Navigate to the \textbf{``Playground''} channel in the OOC category.
  \item Type \textbf{``?search''} into the message box and press enter. Familiarize yourself with the output that the discord  bot prints in the channel. You can always repeat this command if you forget the name of a sign or container.
	\item Read the sign by typing \textbf{``?readsign Sign 1''} and press enter. Read what the discord bot prints in the channel. Notice that this sign says you can switch it with another sign called ``Sign 2.''
	\item Switch an active and inactive sign by typing \textbf{``?changesign Sign 1 . Sign 2''}. 
	\item Search the area again. Notice that Sign 1 is not listed any more, but Sign 2 is.
	\item Read ``Sign 2'' using the appropriate command.
	\item Switch the signs back again using the appropriate command.
	\item Go to your PRIVATE CHANNEL: type \textbf{``?inventory''} to see your items, abilities, and mem packets (by name). Do NOT react to delete the message.
	\item Return to the ``Playground'' channel.
	\item Read the container by typing \textbf{``?searchcontainer Box''}. Read what the discord bot prints in the channel. React \textbf{``X''} to end the interaction and delete the message.
	\item Read the container again. This time search the container by reacting \textbf{``Check mark''}.
	\item Wait for the search to occur and the bot to start printing items. Different containers may have different search times; this Box has a 10 second search time. It will take a few seconds for the bot to print everything once it starts. You will know it is done when it prints the final option ``React here to not take any items.''
	\item React with the \textbf{``check mark''} to one of the blobs to take it.
	\item Go back to your PRIVATE CHANNEL: Check your inventory using the appropriate command. Compare the item list from before to the one you have now. You should see the blob you just took. 
	\item Check the details of the Blob by typing ``?listitem Blob'' in your private channel. This is how you can read the description of items you have. Please try to do this with each item you pick up. Some items will have important information in the description that can only be seen by using the ``?listitem'' command.
	\item Return to the ``Playground'' channel.
	\item Put the blob back in the container by typing \textbf{``?drop Blob . Box''}.
	\item Go back to your PRIVATE CHANNEL: Check your inventory using the appropriate command. Compare the item list from before to the one you have now. You should see the blob is gone. Delete the most recent inventory by reacting ``check mark''. Try it with the older messages and notice that it doesn't work if those messages are more than 10 seconds old.
	\item Take a blob from the box again using the appropriate command.
	\item Give the blob to another character who is on the server with you at the same time by typing \textbf{``?give $<$@user$>$ Blob''}. A GM can temporarily take on a character role to facilitate this. NOTE: You cannot tag a role for this to work (i.e: GM, Captain, etc). You can only tag users by the nickname they are using in the server (aka: their character name) or their Discord User-name.
	\item Have them give the blob back to you using the appropriate command.
	\item Consume your blob by typing \textbf{``?consumeitem Blob''}. This will remove the blob from game. If you check your inventory one more time, you'll see that the blob is gone.
\end{enumerate}

\end{document}
